\RequirePackage{etex}

% Vi benytter 2 sidet, a4 størrelse og dansk sprog. 
\documentclass[11pt,a4paper,oneside,danish]{memoir}
\usepackage[paper=A4,pagesize]{typearea}
 \usepackage[utf8]{inputenc} % danske tegn

% Choose language by uncommenting danish or english
\usepackage[danish]{babel} % dansk sprogpakke
% \usepackage[english]{babel}


%% Pakker til fonts, farver, referencer og ligenende er placeret relevant sted i dokumentet.
     
     %% FONT
     \usepackage[default,osfigures,scale=0.95]{opensans} % Alternatively use the option 'defaultsans' instead of 'default' to replace the sans serif font only.
     \usepackage[T1]{fontenc}
     
     % Common packages 
     \usepackage{footnote} % Package used to controle footnote(specially in tables etc.)
     \usepackage{amsmath}
     \usepackage{slantsc}
     \usepackage{multirow} % Merge table rows
     \usepackage{longtable} % Table that can span multiple pages
     \usepackage{geometry}
     \usepackage[section]{placeins} % \FloatBarrier
     \usepackage[compatibility=false]{caption}
     \captionsetup{font=small,labelfont=bf,textfont=it}
     \usepackage{graphicx}
     \usepackage{afterpage}
     \usepackage[draft]{fixme} %hvor 'options' bl.a. inkluderer final, draft, danish, inline og margin
     \usepackage{rotating}
     \usepackage{eso-pic} % til inds�ttelse af forside.
     \usepackage{rotating} % rotate stuff(EVERYTHING!11)

	% % MATH STUFF 
    %\usepackage{algorithmic}
    %\usepackage{harpoon} %Tilf�jer harpun-pile over st�rrelser i math, m.m.
    %\usepackage{pst-plot} % For axes   
    %\usepackage{breqn}
    %\usepackage{fourier}  %................... Roman+math - Utopia
    
  
    %\usepackage{nomentbl} % http://ctan.mackichan.com/macros/latex/contrib/nomentbl/nomentbl.pdf

    %\usepackage{psfrag} % insæt tag, mest brugt i forhold til grafer

%%% DIVERSE CUSTOMS PAKKET NED I FILER FOR OVERBLIK
   % \input{styles/HaxTilPreamble.tex}
   % \input{styles/CustomFunctions.tex}
%	\input{Styles/code.tex} 
    \usepackage{todonotes}


% new commands for individually colored todo 
\newcommand{\todosander}[2]{
\todo[#1, color=green!40]{#2 \\- Sander}
}
\newcommand{\todosimon}[2]{
\todo[#1, color=red!40]{#2 \\- Simon}
}
\newcommand{\todorasmus}[2]{
\todo[#1, color=yellow!40]{#2 \\- Rasmus}
}
\newcommand{\todokristoffer}[2]{
\todo[#1, color=purple!40]{#2 \\- Simon}
}


% Color package. Use with listings package and use for color a table
%\usepackage[usenames,dvipsnames,svgnames,table]{xcolor}

    \input{styles/figure.tex}
    
    % add a box using the themecolor. Can be used to wrap around a chapintro for introducing an assignment
    %use \begin{assignment} and end\{assignment} to box it
    %http://texdoc.net/texmf-dist/doc/latex/mdframed/mdframed.pdf
     
    \usepackage[xcolor]{mdframed} 
    \newmdenv[linecolor=ThemeColor]{assignment}
%    \usepackage{Placeins}
    
% % % % % % % % % % % % % % ********% % %Generelt design % % % % % % % % % % % % % % % % % % %
 
%\usepackage[table,dvipsnames]{xcolor} % 'table' er i stedet for \usepackage{colortbl} og 'dvipsnames' er et farveskema hvor man angiver farver med navne

\usepackage{xcolor} % farver kan indsætttes i henhold til RGB med definecolor

%Valg af temafarve % Denne farve bruges i chapterstyle %
\definecolor{ThemeColor}{RGB}{209, 67, 26}

% Valg af font %
\usepackage[sc]{mathpazo}
\linespread{1.05}         % Palatino needs more leading (space between lines)
\usepackage[T1]{fontenc}

% Pakker inkludere overordnede designvalg. Pakkerne er valgt for overskuelighed %
	 \input{Styles/chapterstyle.tex}
	 \input{Styles/header.tex}
	 
% Setup the margins:
\setlrmarginsandblock{1in}{*}{1.5}
\setulmarginsandblock{3cm}{4cm}{*}

% Setup header and footer: %
% \setheadfoot{headheight}{footskip}
% \setheaderspaces{headdrop}{headsep}{ratio}
\setheadfoot{45pt}{1cm}
\setheaderspaces{1cm}{*}{*}
\checkandfixthelayout
% Se dette pdf for dokmentaton:
% www.tug.dk/_media/foredrag/2003-12-03/slides-src.pdf

 \checkandfixthelayout[nearest]
    \checkthelayout
    \fixthelayout
    \baselineskip \topskip

%BibTex setup
\usepackage[semicolon,authoryear,square]{natbib}

\bibliographystyle{plainnat}
%%%%%%%%%%%%%%%%%%%%%%%%%%%%%%%%%%%%%%%%%%%%%%%%%%%%%%%%%%%%%%%%%%%%%%%%%%%%%%%%%%%%%%%%%%%%%%

% % Smart listing function. Documentation: http://texblog.org/tag/makenomenclature/
\usepackage{nomencl}
\makenomenclature 

% % % % % % % % % % % % % % % % Reference opsætning % % % % % % % % % % % % % % % % % % % % % % %

\usepackage{varioref} % for references, http://www.ctex.org/documents/packages/bibref/varioref.pdf
   
% External reference package, zref makes the references into hyperlinks.
\usepackage{url}
\usepackage{hyperref}
%\usepackage{zref-xr}
\usepackage{xr}

% % % % % % % % % % % % % % % % % % % % % % % % % % % % % % % % % % % % % % % % % % % % % % % % % % %

%\usepackage{showlabels}

% footruleskip is allready defined in memoir, so we have to undefine it, in order to include fancyhdr for our footer and header.
\let\footruleskip\undefined
\usepackage{fancyhdr}% For making fancy headers http://ctan.org/pkg/fancyhdr

\AtBeginDocument{\renewcommand\contentsname{Indholdsfortegnelse}}

% % % % % % % % % % % % % Inkluderinger af ønskede kapitler % % % % % % % % % % % % %

\begin{document}
\newgeometry{left=3cm,bottom=0.1cm}

\centerline{\Huge\huge\bfseries\color{ThemeColor}Inversion of Control} 
\vspace{1em}
\centerline{\large\color{ThemeColor}Aarhus School of Engineering - Aarhus Universitet}


\centerline{\large\date{10. April 2015}}

%\color{Black}
%\vspace{8em} % Tilføj nogle tomme linjer
\begin{center}
    \begin{tabular}{ l l p{6cm} }
	 \textbf{Gruppe 11}& & \\
	& & \\
    Stud. nr.: 201270115 & Johnny Davidsen Poulsen & \\\hline
    & & \\
    Stud. nr.: 201271039  & Kristian Rossen Kristensen & \\\hline
     \end{tabular}
      	
\end{center}
\thispagestyle{empty} % No header and footer on this page
\restoregeometry
\include{Kapitler/Indholdsfortegnelse}

% Setup header and footer
\pagestyle{fancy}
\fancyhf{} % Clear all header an footer fields 
\renewcommand{\headrulewidth}{0pt} % Delete headruler
\fancyhead[R]{\includegraphics[height=40pt]{header}} % Insert picture in left side of header
%\fancyhead[R]{\rightmark} % insert chapter title in right side of header
\fancyfoot[LE,RO]{\thepage} % insert pagenumber on right side of even page, left side of odd page 
%\setcounter{page}{1}

\thispagestyle{fancy}
\chapter{Beskrivelse}
\label{chp:description}

\section{Indhold}
Inversion of Control er et design pattern som 
Dependency injection design pattern er 

Uafhængighed mellem klasser

Eventdrevet instantiering af objekter

Frameworks


\section{Formål}

Skabe uafhængighed mellem klasser

\section{Konsekvenser}


\thispagestyle{fancy}
\chapter{Sammenligning}
\label{chp:comparison}
\todo[inline]{Sammenligne emnet med andre, tilsvarende emner. Disse maa gerne have vaeret behandlet i undervisningen.}

\section{Sektion}
\thispagestyle{fancy}
\chapter{Eksempler}
\label{chp:examples}
\todo[inline]{Eksemplificere emnet – for eksempel vise en implementering af emnet, et eksempel-gennemløb af udviklingsprocessen eller lignende.}


\thispagestyle{fancy}
\chapter{Konklusion}
\label{chp:Konklusion}
\todo[inline]{Konkludere paa emnet – hvornår giver det bedst mening at anvende emnet, og i hvilke(n) situation(er) giver det ingen mening?}

%\bibliographystyle{plainnat}
%\bibliography{bibliografi}% Selects .bib file AND prints bibliography
\thispagestyle{fancy}

%\listoftodos % Make a list of todo's
\end{document}