\thispagestyle{fancy}
\chapter{Konklusion}
\label{chp:Konklusion}
IoC giver mest mening i eventdrevne systemer hvor ansvaret for instantiering af klasser ikke nødvendigvis kan placeres i egne klasser. Med IoC Containers oprettes objekter når der er brug for dem og nedlægges når de ikke længere er nødvendige. 
IoC containers skaber lav afhængighed mellem klasser og gør det nemmere at opfylde Open/close og dependency inversion principperne.
Den lave lave kopling gør det desuden nemmere at genbruge klasser i andre applikationer da de ikke er bundet til en custom factory.
IoC kan påvirke et programs performance, og er muligvis ikke det bedste valg i mindre ikke-eventdrevne applikationer. 